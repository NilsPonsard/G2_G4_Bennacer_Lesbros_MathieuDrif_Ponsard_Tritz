\documentclass{article}

\usepackage{array}
\usepackage[utf8]{inputenc}
\usepackage[T1]{fontenc}
\usepackage[french]{babel}
\usepackage{amsmath}
\usepackage[a4paper, total={6in, 8in}]{geometry}

\raggedright
\begin{document}
\author{Nils PONSARD \and Clément MATHIEU-DRIF \and
Hugo TRITZ \and BENNACER Chakib \and
Raphaël LESBROS}
\title{compte rendu projet cpp}
\maketitle
\paragraph{} Nous avons décidé de faire le jeu Space Invader en utilisant minGL pour le rendre plus agréable à jouer et pour qu'il ressemble plus au jeu original
\section{Ajouts au jeu}
\begin{itemize}
    \item Ajout de musique et de bruitage 
    \item Ajout d'une page d'accueil avant de démarrer le jeu 
    \item Ajout d'un menu de pause permettant de quitter ou de reprendre le jeu
    \item Ajout d'un système de niveaux qui rajoute des invaders et qui leur donne plus de vie au cours des niveaux 
    \item Ajout de powerUps donnant différents bonus : Vie supplémentaire, Bonus de score.
    \item Ajout d'un système de score augmentant à chaque invader détruit ou à chaque roquette détruite.
    \item Ajout d'un système de vie, on perd une vie quand on se fait toucher
\end{itemize}
\paragraph{} Ces fonctionnalités ont été testé en les utilisant dans le jeu, en adaptant les paramètres pour faciliter les tests.
\section{Gestion des paramètres}
\paragraph{} Les paramètres sont stockés dans config.yaml, il utilise le format de paramètres yaml.
Si le fichier n'existe pas il est créé avec les valeurs par défaut. Si une valeur n'est pas présente ou incorrecte dans le fichier de configuration, la valeur par défaut est chargée.
\paragraph{} fonctionnalité testé en supprimant le fichier de configuration, en le modifiant avec des valeurs incorrectes ou manquantes. 
\section{Enregistrement des scores}
\paragraph{} Quand un nouveau meilleur score est enregistré, votre nom (en trois lettres) est demandé. Ce score est enregistré dans le fichier scores.yaml qui est créé si inexistant. Le fichier contient max 3 scores triés dans l'ordre décroissant. Si une valeur incorrecte est présente, la ligne est supprimée et le score de la prochaine partie sera energistré. L'insertion d'un score est faite pour que la liste des scores soit toujours triée dans l'ordre décroissant.
\paragraph{} Cette fonction a été testé en jouant et en battant les meilleurs scores, en supprimant le fichier des scores et en mettant des valeurs incorrectes.
\section{Système de collisions}
\paragraph{}L'écran de minGL étant composé de pixel et pas de caractères, nous avons fait une fonction pour tester la collision entre deux carrés avec des positions et des tailles en pixels. Cette fonction a été testée avec la fonction testCollision() dans le dossier testsFonctions.
\section{Modifications de minGL}
\paragraph{} Nous avons modifié le code source de minGL pour pouvoir faire un jeu interactif, ces modifications sont :
\begin{itemize}
    \item Ajout de la fonction displayText permettant d'afficher du texte à l'écran, testé en l'intégrant au jeu et en le lançant.
    \item Ajout de la fonction setBgColor permettant de changer la couleur d'arrière-plan .
    \item Modification de la gestion des touches pour savoir quelles touches sont pressées sans interompre l'éxécution du jeu.
    \item Modification de la gestion de l'affichage (updateGraphic et callDisplay) pour enlever un le clignotement du texte.
    \item Ajout de la ligne glutSetKeyRepeat(GLUT\_KEY\_REPEAT\_OFF) qui désactive la répétition de touches pour une meilleure navigation dans les menus.
    \item Ajout de la fonction ResetKey() permettant de réinitialiser l'état de la touche à false (pas pressée)
\end{itemize}



\end{document}
