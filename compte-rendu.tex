\documentclass{article}

\usepackage{array}
\usepackage[utf8]{inputenc}
\usepackage[T1]{fontenc}
\usepackage[french]{babel}
\usepackage{amsmath}
\usepackage[a4paper, total={6in, 8in}]{geometry}

\raggedright
\begin{document}
\author{Nils PONSARD \and Clément MATHIEU-DRIF \and
Hugo TRITZ \and BENNACER Chakib \and
Raphaël LESBROS}
\title{compte rendu projet cpp}
\maketitle
\section{Modifications de minGL}
\paragraph{} Nous avons modifié le code source de minGL pour pouvoir faire un jeu interactif, ces modifications sont :
\begin{itemize}
    \item Ajout de la fonction displayText permettant d'afficher du texte à l'écran, testé en l'intégrant au jeu et en le lançant.
    \item Ajout de la fonction setBgColor permettant de changer la couleur d'arrière-plan .
    \item Modification de la gestion des touches pour savoir quelles touches sont pressées sans interompre l'éxécution du jeu.
    \item Modification de la gestion de l'affichage (updateGraphic et callDisplay) pour enlever un le clignotement du texte.
    \item 
    \item Ajout de la ligne glutSetKeyRepeat(GLUT\_KEY\_REPEAT\_OFF) qui désactive la répétition de touches pour une meilleure navigation dans les menus.
\end{itemize}



\end{document}